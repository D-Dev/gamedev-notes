\section{18.01 - Single Variable Calculus}

\begin{featurebox}
David Jerison. 18.01SC Single Variable Calculus. Fall 2010. Massachusetts Institute of Technology: MIT OpenCourseWare, https://ocw.mit.edu. License: Creative Commons BY-NC-SA.
\\\\
This calculus course covers differentiation and integration of functions of one variable, and concludes with a brief discussion of infinite series. Calculus is fundamental to many scientific disciplines including physics, engineering, and economics.
\end{featurebox}
\newpage

\subsection{Unit 1: Differentiation}




\newpage

\subsubsection{Part A: Definition and Basic Rules}

Geometric Interpretation of differentiation - find the tangent line to $y=f(x)$ at $\ubar{P} = (x_0, y_0)$.

\begin{definition}
$f'(x_0)$, the derivative of $f$ at $x_0$, is the slope of the tangent line to $y = f(x)$ at $\ubar{P}$.
\end{definition}

\begin{definition}Tangent Line is the limit of secant lines $PQ$ as $Q \to P$ where $P$ is fixed.
\end{definition}


\begin{figure}[h]
\caption{Main Formula}
\[f'(x_0) = \lim_{x \to 0}\frac{f(x_0) + \Delta x) - f(x_0)}{\Delta x}\]

\end{figure}

\newpage

\subsubsection{Problem Set 1}

\begin{comment}
\newpage

\subsubsection{Part B: Implicit Differentiation and Inverse Functions}

\newpage

\subsubsection{Problem Set 2}

\newpage

\subsubsection{Exam 1}

\newpage

\subsection{Unit 2: Applications of Differentiation}

\newpage

\subsubsection{Part A: Approximation and Curve Sketching}

\newpage

\subsubsection{Problem Set 3}

\newpage

\subsubsection{Part B: Optimization, Related Rates and Newton's Method}

\newpage

\subsubsection{Problem Set 4}

\newpage

\subsubsection{Part C: Mean Value Theorem, Antiderivatives and Differential Equations}

\newpage

\subsubsection{Problem Set 5}

\newpage

\subsubsection{Exam 2}

\newpage

\subsection{Unit 3: The Definite Integral and its Applications}

\newpage

\subsubsection{Part A: Definition of the Definite Integral and First Fundamental Theorem}

\newpage

\subsubsection{Problem Set 6}

\newpage

\subsubsection{Part B: Second Fundamental Theorem, Areas, Volumes}

\newpage

\subsubsection{Problem Set 7}

\newpage

\subsubsection{Part C: Average Value, Probability and Numerical Integration}

\newpage

\subsubsection{Problem Set 8}

\newpage

\subsubsection{Exam 3}

\newpage

\subsection{Unit 4: Techniques of Integration}

\newpage

\subsubsection{Part A: Trigonometric Powers, Trigonometric Substitution and Completing the Square}

\newpage

\subsubsection{Problem Set 9}

\newpage

\subsubsection{Part A: Trigonometric Powers, Trigonometric Substitution and Completing the Square}

\newpage

\subsubsection{Problem Set 10}

\newpage

\subsubsection{Part C: Parametric Equations and Polar Coordinates}

\newpage

\subsubsection{Problem Set 1}

\newpage

\subsubsection{Exam 4}

\newpage

\subsection{Unit 5: Exploring the Infinite}

\newpage

\subsubsection{Part A: L'Hospital's Rule and Improper Integrals}

\newpage

\subsubsection{Part B: Taylor Series}
\end{comment}
\newpage

